\documentclass{article}
\usepackage[utf8]{inputenc}
\usepackage{amssymb}
\title{20 proofs in 1 sitting}
\author{J.J.K Groenendijk}
\date{26/10/2020}
\begin{document}

\section{Exam}

1. A y\\
2. C y\\
3. B y\\
4. A y\\
5. B y\\
6. B y\\
7. B y\\
8. A y\\
9. A y\\
10. C y\\
11. C y\\
12. C y\\
13. D y\\
14. C y\\
15. B y\\
16. B y\\
17. B y\\
18. C y\\
19. D y\\
20. D y\\
21. \\
 a. \\
[insert truth table] y
 b. No, becuase the row [x] has a different value for both / Yes because every row underneith the final columns has the same value.y
 
 \\c. Take all values that are false. Then do \neg((row[p] \land row[q] \land row[r]) \lor (nextrow[p] \land nextrow[q] \land nextrow[r]) \lor ...) and then simplify out using deMorgans law until at CNF. 
 y
22. \\
 a.\\
 D = {0,1,2,3,4,5,6,7,8,9,10}\\
 E = {0,2,4,6,8,10}\\
 P = {2,3,5,7}\\
 S = {0,1,4,9}\\
 R = {(9,10),(10,9),(8,10),(10,8),(9,9),(10,10)}\\
 b. Yes, namely x=2\\
 \\
 \\
23.
a.
To prove: $$\forall n \geq 1, n \in N: \sum_{i=1}^{n}i(i+1) = \frac{n(n+1)(n+2)}{3}$$
basecase: for $n=1$, $$\sum_{i=1}^{1}i(i+1) = \frac{1(1+1)(1+2)}{3}$$ 
$$1(1+1) = \frac{6}{3}$$
$$2 = 2$$
The base case holds.

I.H: We assume, for an arbitrary $n \in N,n \geq 1$:
$$\sum_{i=1}^{n}i(i+1) = \frac{n(n+1)(n+2)}{3}$$\\
We want to prove: $$\sum_{i=1}^{n}i(i+1) = \frac{n(n+1)(n+2)}{3} \rightarrow \sum_{i=1}^{n+1}i(i+1) = \frac{(n+1)(n+1+1)(n+2+1)}{3}$$

$$\sum_{i=1}^{n+1}i(i+1) = \frac{(n+1)(n+1+1)(n+2+1)}{3}$$
$$\sum_{i=1}^{n+1}i(i+1) = \frac{(n+1)(n+2)(n+3)}{3}$$
$$(n+1)(n+2) + \sum_{i=1}^{n}i(i+1) = \frac{(n+1)(n+2)(n+3)}{3}$$
By the I.H.:
$$(n+1)(n+2) + \frac{n(n+1)(n+2)}{3} = \frac{(n+1)(n+2)(n+3)}{3}$$
$$\frac{3(n+1)(n+2)}{3} + \frac{n(n+1)(n+2)}{3} = \frac{(n+1)(n+2)(n+3)}{3}$$
$$\frac{3(n+1)(n+2) + n(n+1)(n+2)}{3} = \frac{(n+1)(n+2)(n+3)}{3}$$
$$\frac{(3+n) (n+1)(n+2)}{3} = \frac{(n+1)(n+2)(n+3)}{3}$$
$$\frac{(n+1)(n+2)(n+3)}{3} = \frac{(n+1)(n+2)(n+3)}{3}$$
Which is what we wanted to prove. n was arbitrarily chosen, therefore, our Proof by Induction is finished, and we have proven that $$\forall n \geq 1, n \in N: \sum_{i=1}^{n}i(i+1) = \frac{n(n+1)(n+2)}{3}$$
$\Box$
y
b. i.\\ 
Okay.\\
$$x = \sum_{i=0}^{0}i$$
$$x = 0$$, which is true.
ii.\\
It holds before the loop, so $$x = \sum_{i=0}^{c}i$$
After the loop:$$x+(c+1) = \sum_{i=0}^{c+1}i$$
$$x+(c+1) = c+1 +\sum_{i=0}^{c}i$$
$$x+(c+1) = c+1 + x$$, which is correct.\\
y
iii. n is a finite number. c starts at 0 and always grows, so at some point, c will be bigger than n.\\
\\
24.\\
a. 
We can rewrite the first part as:\\
$(x \in A \oplus x \in B) \land (x \in C)$\\
$((x \in A \lor x \in B) \land \neg(x \in A \land x \in B)) \land (x \in C)$\\
$((x \in A \lor x \in B) \land ( x \not \in A \lor  x \not\in B)) \land (x \in C)$\\
\\
\\



We can rewrite the second part as:\\
$(x \in A \land x \in C) \oplus (x \in B \land x \in C)$\\
$((x \in A \land x \in C) \lor (x \in B \land x \in C)) \land \neg((x \in A \land x \in C) \land (x \in B \land x \in C))$\\
$((x \in A \lor x \in B ) \land x \in C)) \land \neg((x \in A \land x \in C) \land (x \in B \land x \in C))$\\
$(x \in A \lor x \in B ) \land x \in C \land (\neg(x \in A \land x \in C) \lor \neg(x \in B \land x \in C))$\\
$(x \in A \lor x \in B ) \land x \in C \land ((x \not \in A \lor x \not \in C) \lor (x \not \in B \lor x \not \in C))$\\
$(x \in A \lor x \in B ) \land x \in C \land (x \not \in A \lor x \not \in B \lor x \not \in C)$\\
$(x \in A \lor x \in B ) \land x \in C \land (x \not \in A \lor x \not \in B)$\\
$(x \in A \lor x \in B ) \land (x \not \in A \lor x \not \in B) \land x \in C $\\
\\

And therefore, we can see that these two statements are the same, which is what we wanted to prove. $\Box$\\
y

b is not true, with as counterexample:\\
A = {1}\\
B = {1,2}\\
C = {2}\\
y
25.\\
a. \\
I : $120 \in S$\\
II: $(x \cdot y = z \land z \in S) \rightarrow x,y \in S$\\
III: $(x,y \in S) \rightarrow (x \cdot y \in S)$\\
Iv: There is nothing else in S.\\
\\
b.\\
Define a function f(x) that takes in a word x and returns the amount of a in that word. To prove: $x \in S \rightarrow 2 \nmid f(x) $\\
We will use a proof by structural induction.\\
Base case: $a \in S$. $f(a) = 1$, which is odd, and therefore the base case holds.\\
\\
Induction hypothesis: $x \in S \land 2 \nmid f(x)$
We assume this is true for some arbitrary $x \in S$.\\

Inductive step 1: $x \in S \rightarrow xi \in S$\\
We know that since $x \in S$, $f(x)=2y+1$ for some integer y.
$f(xi) = f(x)+f(i) = 2y+1 + 0 = 2y+1$, which is odd, which is what we wanted to prove.\\
\\

Inductive step 2: $x \in S \rightarrow axa \in S$\\
We know that since $x \in S$, $f(x)=2y+1$ for some integer y.
$f(axa) = f(a)+f(x)+f(a) = 1+ 2y+1 + 1 = 2(y+1)+1$, which is odd, which is what we wanted to prove.

Inductive step 3: $x,y \in S \rightarrow ixiyixi \in S$\\

We know that since $x \in S$, $f(x)=2a+1$ for some integer a, and similarly, since $y \in S$, $f(y)=2b+1$ for some integer b.\\

$f(ixiyixi) = f(i)+f(x)+f(i)+f(y)+f(i)+f(x)+f(i) = 0+ 2a+1 + 0 + 2b+1 + 0 + 2a+1 + 0 = 2(2a+b+1)+1$, which is odd, which is what we wanted to prove.\\
\\
And since we have proven this all for some arbitrary element x in S, this concludes our proof by structural induction and we have proven that if x is in S, then the number of a in x is odd.\\
$\Box$\\
\\
26.a\\
Every function is a relation, but not every relation is a function. For example, $f(X) = \pm x$ is a relation but not a function. This is because functions can only have 1 output per input, while relations do not have that constraint.\\

b.\\
i. This one does, namely $f(x) = \frac{x-26}{18}$\\
ii. No, the cardinality of Z and R are different, they are not binjective, and therefore they do not have an inverse.
iii. Sure, $h^{-1} = \{(b,a),(d,c),(f,e),(h,g),(j,i)\}$\\
iv. Sure, namely $l^{-1} = \{(1,1),(2,2),(3,3)\}$\\
c. \\
i. is true, A triangle is always similar to itself, if triangle a is similar to triangle b, and triangle b is similar to triangle c, then triangle a is similar to triangle c. And lastly, if triangle a is similar to triangle b, then triangle b is similar to triangle a. Similarity has all the properties of an equivalence relationship, and therefore it is an equivalence relationship. \\
\\
ii. suppose there is a person x, with an age of 1 and a wage of 2, and there is person y with an age of 100 and a wage of 1. \\
First, we can deduce (x,y) is in b, because
3<=101. \\
To be an equivalence relationship, (x,x) should be in B. Therefore, 2+2<=1+1, so 4<=2. However, this does not hold, so B is not an equivalence relationship. 

\end {document}